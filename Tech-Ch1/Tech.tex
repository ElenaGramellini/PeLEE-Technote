Before we dive into the physics, we take a moment to layout a some technical details, common conventions and acronyms used throughout the note, with the hope to help the reader through the note's interpretation.
%\addcontentsline{toc}{Section}{echnical Overview on the Content of this Note}
\subsection{Samples}
\par This analysis attempts to use all the latest available data and simulation samples for the MicroBooNE LEE analysis. This section briefly describes what samples are used.
\par MicroBooNE's collected dataset passing  data-quality requirements for the LEE analysis consists of $10.1E20$ POT of BNB data taken over four Runs. %This number takes into account run periods discarded due to data-quality requirements. 
Only small subsets of this data is available for analyzers to use, i.e. :
\begin{itemize}
\item[-] On-beam data (\textbf{BNB}):
\begin{itemize}
\item ``5E19'' POT Run1 open dataset (4.07E19 POT after quality cuts)
\item  ``1E19'' POT Run3 open dataset  (0.8E19 POT after quality cuts) 
\item  $\sim3.0E20$ POT of filtered $\pi^0$ data from Runs 1 and 3. 
\end{itemize}
\item[-] Off-beam data (\textbf{EXT}):
\begin{itemize}
\item all available Run1 samples (about 1 million events)
\item all available Run3 samples (about 1 million events)
\end{itemize}
\item[-] MC (from the ``overlay datasets''):
\begin{itemize}
\item $\sim$1E21 POT standard MC BNB flux prediction
\item $\sim$5E22 POT $\nu_e$ intrinsic only from the standard MicroBooNE flux prediction
\item $\sim$3E20 POT  \emph{dirt} interactions (neutrinos which interact outside of the cryostat) 
\item $\sim$2-3E21 POT filtered by final-state topology for CC $\pi^0$ and NC $\pi^0$ events
\end{itemize}
\end{itemize}

%For on-beam data, three samples are used in this note: the ``5E19'' Run1 open dataset ($4.07E19$ POT after quality cuts), the ``1E19'' ($0.8E19$ POT after quality cuts) open Run3 data, and the $\sim3.0E20$ POT of filtered $\pi^0$ data from runs 1 and 3. 
%For off-beam data, this analysis uses all available samples from Run1 and Run3, consisting of roughly 1 million events for each run. 

Scaling factors for the EXT datasets to match the full $10.1E20$ POT are $\times3.5$ and $\times2.1$ for Run1 and Run3 respectively. The fact that EXT samples need to be scaled up rather then down is a consequence of the lack of availability of off-beam events, a generally recognized problem with all LEE analyses which may make EXT background estimation difficult and hamper sensitivity estimates.
\par %All MC samples used are from ``overlay datasets''. For each run period, MC datasets are produced for the standard MC BNB flux prediction (``MC'' sample, $\sim1E21$ POT), a $\nu_e$ intrinsic sample simulating only $\nu_e$ neutrinos from the standard MicroBooNE flux prediction (\emph{$\nu_e$ intrinsic}, $\sim5E22$ POT), a sample of \emph{dirt} interactions (neutrinos which interact outside of the cryostat, $\sim3E20$ POT), and samples filtered by final-state topology for CC $\pi^0$ and NC $\pi^0$ channels ($\sim2-3E21$ POT). 
All simulation samples in addition to the generic ``MC'' are used to boost statistics in channels that are sub-dominant in event rate but important for this analysis. In merging these samples in the analysis, we take care to remove the same events from the ``MC'' samples and weigh the high-stats special samples appropriately. 
Even with this workflow, the statistics of several background categories is not sufficient for a careful evaluation of their impact on the selection, leading to large bin-by-bin fluctuations in expected events and therefore an unclear background estimation. To mitigate this, specific targeted truth-based filters have been developed (see DocDB 27409~\cite{bib:truthfilters}) to boost statistics in backgrounds for which current statistics are not sufficient. 

\subsection{How to read plots in this tech-note} A large fraction of plots in this note are data-simulation comparison. This paragraph describes common features for these plots. All data-simulation comparisons up to Chapter~\ref{sec:systematics} show statistical errors only, denoted with error-bars for data. The statistical error on MC samples is denoted with a grey band. All plots, unless otherwise labeled, are POT normalized, following the procedure outlined in DocDB 15204~\cite{bib:POTscaling}. Data-simulation comparisons are accompanied by a legend which denotes the various event categories shown. Categories are described below:
\begin{itemize}
    \item \textbf{BNB}: On-beam contributions (data).
    \item \textbf{EXT}: Off-beam contributions (data).
    \item \textbf{Cosmic}: Events from MC overlay where a cosmic instead of a neutrino interaction have been selected as the neutrino candidate by Pandora's \texttt{SliceID}.
    \item \textbf{Out. fid. vol.}: Events from MC overlay where a neutrino was tagged, but the neutrino interaction vertex is outside of the defined TPC fiducial volume.
    \item \textbf{MiniBooNE LEE}: Events from the MB-$\nu_e$ LEE model.
    \item \textbf{$\nu_e$}: CC $\nu_e$ events, split by final state (\zpsel, \npsel, and N$\pi$).
    \item \textbf{$\nu_{\mu}$ CC}: CC muon events with no final state $\pi^0$.
    \item \textbf{$\pi^0$}: $\nu_{\mu}$ events with final state $\pi^0$ split in NC and CC channels.
    \item \textbf{NC}: NC events (all flavors) with no final state $\pi^0$s.
\end{itemize}{}

\subsection{Tech-note versioning: current status and recent/expected changes }
Until Neutrino2020, this note is supposed to be a living document: this entails that parts of the analysis will evolve and parts of the text is going to change. Throughout the note, we highlight incomplete parts and known preliminary statements with the use of \emph{italics}.

\par As of January 2020,  all latest samples are used, including the ``GENIE-TUNE CV" weights for work up to the selections development.
For systematic studies, the latest GENIE and flux systematics are used to present truth-level studies, but they are not used for the final sensitivity calculation due to ongoing work in SBNFit to incorporate them.
%\begin{multicols}{1}
